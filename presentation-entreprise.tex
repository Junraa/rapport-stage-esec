\section*{L'entreprise et son secteur d'activité}
\paragraph{Sogeti et l'ESEC}
\subparagraph{Sogeti}
\begin{center}
\includegraphics[scale=0.7]{sogeti.png}
\end{center}
Sogeti est une Entreprise de Service du Numérique (ESN), filliale appartenant
à 100\% à Capgemini. Fondée en 1967 à Grenoble par Serge Kampf, elle opère
\paragraph{}
\paragraph{}

\subparagraph{European Security Expertise Center}
\begin{center}
\includegraphics[scale=0.4]{esec.png}
\end{center}
L'European Security Expertise Center (ESEC), est le pôle responsable des domaines
lié à la sécurité de Sogeti. Le centre est composé plusieurs équipes, dont un
laboratoire de recherche et developpement, ou une équipe dédiée aux tests de penetration.
C'est dans cette dernière équipe que le stage s'est déroulé.

L'ESEC est directement placé sous le pôle sécurité de Capgemini, dirigé par Frank
Greverie, secondé par Jean Marc Bianchini pour Sogeti.

Au niveau France, Sogeti et l'ESEC ont plusieurs concurrents, dont Solucom, Deloitte,
etc. Le domaine de la sécurité informatique étant en explosion, de nombreuses ESN existante
essayent de développer leurs équipes dans le domaine, afin de gagner des parts de marché
sur des sujets de plus en plus demandeur en terme de force salariale.


\paragraph{}
\subparagraph{}
\section*{Le service: Test de pénétration / Pentest}

\subparagraph{}

\section*{Etat des connaissances avant le stage sur le sujet}
\paragraph{}
Le cursus de la majeure SRS au sein de l'EPITA a pour but d'apporter une base technique et fonctionnelle
sur la sécurité. En correlation avec le niveau général de la promotion, les points abordés peuvent être
plus ou moins avancées. Le sujet de stage portait sur de l'analyse statique de programme binaire, sans rapports
d'outils externes, pour la detection de vulnérabilité de type usage-après-libération (use-after-free).
\subparagraph{}
Les prè-requis sont donc:
\begin{itemize}
\item L'analyse statique instrumentalisée.
\item La compréhension du fonctionnement des allocateurs de mémoires.
\item Le principe de tainte de mémoire/registre pour l'analyse statique.
\item Des notions en théorie des graphes.
\item Des notions en rétro-ingéniérie.
\end{itemize}

De ces points, la rétro-ingéniérie et la théorie des graphes sont suffisament enseignés à l'EPITA,
durant l'année de spécialisation ou bien pendant la première année du cycle ingénieur.
\subparagraph{}
Concernant les autres points, l'analyse statique instrumentalisée, en comparaison avec ce qui a été
fait durant le stage, n'était après retour pas très avancée, et les bases étaient difficilement acquises par
l'ensemble de la promotion SRS. Même si le thème de l'analyse statique était bien abordé, le domaine de l'instrumentalisation
aurait peut être eu besoin de plus de temps pour être enseigné et compris. Le principe de tainte mémoire/registre étant intimement
lié à ce dernier point, les commentaires précédents s'y appliquent également.
\subparagraph{}
Le point majoritairement manquant concernant ce stage et la plus grosse lacune concernant le sujet du stage résidait dans la compréhension de
la gestion de la mémoire par un système d'exploitation, des différents niveaux d'allocation mémoire, et de leur fonctionnement précis
selon les cas (niveau noyau ou niveau utilisateur, différents appels, granularité, algorithme de défragmentation/optimisation, ...).
\subparagraph{}
Cette notion est vraiment très peu enseignée durant le cursus de l'EPITA, quelque soit la spécialisation, même si une partie du projet KSTOS
demandait (à un niveau relativement peu avancé) la compréhension de la relation mémoire physique / mémoire virtuelle.
Les notions et l'études des ABI (Interface Application-Binaire / Application Binary Interface) est elle aussi assez peu évoquée.
\subparagraph{}

Néanmoins, le premier mois de stage a été dédié aux points defaillants et le reste du stage a pu se dérouler avec assez de connaissance
pour être confortable dans ces domaines là.

\subparagraph{}


\section*{Positionnement du stage dans l'entreprise}
\paragraph{}
\subparagraph{}
L'équipe pentest cherche à atteindre plusieurs objectifs à travers leurs offres de stage.
\subparagraph{}
Le stagiaire doit tout d'abord monter en compétence sur un domaine prècis afin d'atteindre un niveau au plus proche de l'expertise sur son sujet de stage.\newline
Tout au long du stage se déroulent des présentations d'avancement et/ou de sujet précis dans le cadre du stage, de la part des stagiaires ayant avancé dans leur domaine et/où trouver un thème à discutter au sein de l'équipe. Ceci fait parti d'une des attentes de l'équipe envers ses stagiaires: la diffusion de la connaissance acquise.
\subparagraph{}
En complement de ces deux objectifs vient naturellement l'envie de développer des solutions internes pouvant être réutilisées/améliorées à la fin du stage. Selon la difficulté du sujet de stage et/ou des compétences du stagiaire, la solution finale sera plus ou moins utile pour soit l'équipe, soit un nouveau stagiaire.\newline

Concernant le sujet du stage présenté dans ce rapport, le type de vulnérabilité, bien que connu de la majorité de l'équipe, restait assez compliqué à comprendre en détail, notamment de part les connaissances en système et conception d'allocateurs mémoire requises. Avoir un stagiaire dans ce domaine permettait donc d'avoir des rappels / formations sur ces points la lors des diffèrentes présentation.


\paragraph{}
\subparagraph{}
\newpage
