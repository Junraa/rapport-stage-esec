\section{Conclusion générale}
En résumé, malgré un mauvais choix de départ (Miasm), je suis satisfait du travail accompli et des connaissances acquises, qui dépasse de loin mes attentes pour un stage technique dans une ESN.
Le sujet de stage était vaste, et une première idée avait été de travailler sur les vulnérabilités de types dépassement de tampon. Néanmoins lors de la réunion de début de stage, la proposition a été
faite de se pencher sur les vulnérabilités de type usage-apres-libération. Ce choix a porté ses fruits, et je pense avoir beaucoup plus appris que si la première proposition avait été gardée.
\subparagraph{}
Dans l'ensemble, le planning avait bien été défini et a été efficace. La période d'apprentissage d'un mois au début du stage a porté ses fruits et la compréhension a été beaucoup plus rapide
après coup. Les choix techniques, notamment le choix de Miasm, aurait pu être meilleur, et le peu de connaissances dans le fonctionnement interne de Python a également ralenti la progression
lors du développement.
\subparagraph{}
La formation SRS de l'EPITA m'a aidé pour la compréhension durant les phases de retro-ingénierie. Néanmoins, l'analyse statique instrumentalisée n'était quasiment pas abordée, et aurait peut être
méritée une plus grande importance. Les domaines annexes (graphes/optimisations) avaient été acquis durant la première année du cycle ingénieur.

\section{Remerciements}
J'aimerai conclure ce rapport en remerciant Adrien Fay pour son aide, sa disponibilité et pour sa façon d'être toujours très agréable tout au long de ces 6 mois,
Guillaume Meunier pour ses éclairages sur le fonctionnement d'une ESN et des missions associées, ainsi que toute l'équipe du pentest de l'ESEC pour leur accueil très agréable et le temps
pris à répondre à mes questions. Un merci également aux membres de la R\&D pour leurs présentation régulière sur des sujets pointus et leurs réponses à mes interrogations.
