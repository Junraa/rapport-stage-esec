\chapter{Cahier des charges}
\section*{Chronogramme}
\begin{center}
\end{center}
\section{Remise en contexte}
\paragraph{Les vulnérabilités de type Use-After-Free}
\subparagraph{}
La plupart des vulnérabilités enseignées et étudiées dans le cadre scolaire
se rapportent à des fonctionnement souvent statique, ou du moins dont le comportement
est uniquement lié au programme ciblé en lui-même (à son architecture de compilation et/à
son code source). On prendra comme exemple les dépassements de tampons, les problèmes de
formattage des chaines de caractères, etc.\subparagraph{}
Dans l'environnement industriel, et sur les solutions importantes utilisées dans le monde
de l'entreprise, ces vulnérabilités sont de plus en plus rare. Premièrement parce que le temps
faisant son oeuvre, la grande majorité ont été reportées sur les solutions les plus anciennes,
mais également parce que de nombreux outils pouvant les détecter rapidement (parfois dès la compilation)
ont été concus.\subparagraph{}
Les attaquants explorent donc de nouvelles voies, notamment sur des problèmatiques plus dynamiques, et/ou
lié également à l'environnement de production du programme, comme le système d'exploitation ou la bibliothèque utilisée.
En effet, de nombreuses opérations du programme dépendent intrisequement du système d'exploitation sous-jacent et
même si le comportement final reste souvent le même sur les differentes plateformes, les effets colateraux
sur le contexte et le mécanisme d'execution est souvent différent.\subparagraph{}

\subparagraph{L'évolution des vulnérabilités de type use-after-free}
\subparagraph{INSERT HISTOGRAM HERE}
Beaucoup de problème de corruption mémoire sont directement dû à la mauvaise utilisation
du language utilisé pour le developpement. En effet, les languages de programmation tels que C
ou C++ demandent une gestion de la mémoire manuelle (le C++ permet néanmoins une gestion plus ou moins automatisée
avec les nouveaux standards).
\subparagraph{}
Il est donc fréquent de rencontrer des erreurs sur ce point là. Cela arrive
notamment lorsque par un problème de maintenabilité ou de mauvaise conception, la partie du programme devant
libérer une zone mémoire est méconnue/non documentée.\subparagraph{}
Ce n'est cependant pas toujours le cas et les allocateurs de mémoires / ramasse-miettes sont parfois en cause.

\subparagraph{Exemple}
Un exemple très simple et souvent rencontré dans la littérature est le suivant:
\begin {lstlisting}[frame=single]
int main(void)
{
    int *a = malloc(sizeof(int));
    *a = 5;
    free(a);
    int *b = malloc(sizeof(int));
    printf("\%d", *b); // affiche 5
    *b = 7;
    printf("\%d", *b); // affiche 7
    *a = 10;
    printf("\%d", *b); // affiche 10
}
\end{lstlisting}
\subparagraph{}
Sur une distribution Linux classique comme Ubuntu, quelque soit la taille de mots
de l'architecture (32 ou 64 bits), les adresses a et b sont exactement les mêmes.
Cela est directement dû à l'algorithme d'allocation mémoire de la bibliothèque standard
C.\subparagraph{}
En effet lors du premier appel à la fonction \textbf{malloc()}, une première zone est allouée.
Lors de sa libération, la zone libérée est mise en cache, et lors du second appel à \textbf{malloc()},
l'algorithme va retourner la même zone mémoire. Ce comportement ne sera pas détaillée dans cette section, car
il demanderait à lui seul plusieurs pages.\subparagraph{}
Cela va de soi que cet exemple est simpliste, et que les vulnérabilités rencontrées en production sont
rarement aussi visible. Il permet toutefois de bien visualiser l'origine du problème.


\paragraph{Outils existants en dynamique et en statique}
Des outils simplistes ne suffisent pas à détecter la présence d'use-after-free.\newline
Par exemple, si une solution détecte uniquement l'accès à une zone mémoire non allouée, l'exemple
donnée précedemment ne sera pas considéré comme vulnérable, car la zone mémoire est là même pour deux
allocations différentes. Plusieurs possibilités s'ouvrent alors : l'analyse statique, l'analyse dynamique,
et un mélange des deux.
\subparagraph{}
De nombreuses personnes / équipes se sont penchées sur la détection de ces comportements, dynamiquement et
statiquement. Nous parlerons tout d'abord des solutions dynamiques existantes, puis nous aborderons \textbf{les approches
statiques, qui sont directement liées au sujet de ce stage.}

\subparagraph{Kasan: Kernel Adress Sanitizer} est un projet open-source, développée majoritairement
par les acteurs du noyaux Linux ainsi que par Google. Son but est d'ajouter une couche de protection
mémoire au noyau Linux afin de détecter certains corruptions mémoires dans les opérations du noyau, comme
des use-after-free ou des accès hors limites.
https://github.com/google/kasan


\section{Présentation globale du projet}

\section{Résultat à obtenir}


\chapter{Compte-rendu d'activité}
\paragraph{}

\section{Prise en main}
\section{Interprétation et critique des résultats}
\subparagraph{}
