\section*{Sujet du stage}
\paragraph{}
Le sujet de stage était présenté sous l'intitulé : "Recherche statique de vulnérabilités dans des binaires. Recherche d'heuristiques pour déterminer la possibilité
d'exploitation de ces vulnérabilités." Le type de vulnérabilités était laissé vague dans le but que le stagiaire puisse choisir dans un éventail large ce qui
servirait le plus à un apport en compétences.  Dans la finalité, le but principal du stage était d'apporter un niveau d'expertise sur le sujet au stagiaire,
avec une première étape liée à l'état de l'art des différents domaines abordés suivi du développement de l'outil en lui même. Ces connaissances devaient ensuite être
partagées avec le reste de l'équipe. Il était attendu un résultat même
non-final, pouvant être repris par la suite, donc documenté et avec une architecture modulaire. Le programme devait constituer une preuve de concept fonctionnant
sur des binaires avec une taille relativement légère par rapport à la moyenne actuelle du marché professionnel. Il serait ensuite repris pour être finaliser et effectuer
une réelle recherche de vulnérabilités.
